\section{Diskussion}
\label{sec:Diskussion}

\subsection{Wheatstonesche Brücke}

Die bestimmten Widerstände aus den einzelnen Messungen liegen, wenn man die berechnete
Unsicherheit mit einbezieht, dicht aneinander. Die Abweichung von $R_{x2}$ auf $R_{x1}$
und $R_{x3}$ beträgt nur \SI{0.9}{\percent}. Die größte Abweichung eines der einzeln
errechneten Werte auf den Mittelwert ist die von $R_{x2}$ und sie beträgt \SI{0.6}{\percent}.
Das sind sehr genaue Werte.
Dies ist auch auf die verhältnismäßig geringen Unsicherheiten der
der Bauteile zurückzuführen.

\subsection{Kapazitätsmessbrücke}

\paragraph{Ideale Kondensatoren}
Hier sind jeweils die ersten beiden Werte identisch und der Dritte weicht um \SI{0.9}{\percent}
von den Anderen ab.
Aber auch der ist noch im Rahmen des Fehlerintervalls. Somit ist der Mittelwert
auch in guter Übereinstimmung mit den Mittelwerten. Erneut hatten die Bauteile
keine große Unsicherheit.

\paragraph{Reale Kondensatoren}
Bei der Bestimmung der Kapazität des realen Kondensators treten ähnlich große Abweichungen
auf, die allesamt im Rahmen der Messgenauigkeit liegen.

Durch die hohe Unsicherheit von $R_2$ mit \SI{3}{\percent} kann der Widerstand
nur sehr ungenau bestimmt werden. Offensichtlich ist dieses Bauteil also nicht
besonders geeignet um den Widerstand eines Kondensators zu bestimmen.

\subsection{Induktivitätsmessbrücke}

Die Induktivität kann hier nur mit einer Unsicherheit bestimmt werden, die
um eine Größenordung größer ist als die der festen Bauteile. Dies liegt daran, dass
$R_2$ wieder eine hohe relative Unsicherheit hatte. Beim Widerstand ist dies analog
der Fall.

\subsection{Maxwell-Brücke}

Nun wird dieselbe Spule erneut vermessen, nun mit der Maxwellbrücke. Der Wert
$L_{xm} = \SI{41.8(7)}{\milli\henry}$ für
die Induktivität, gemessen an der Maxwell-Brücke, weicht nur um \SI{0.2}{\percent}
von dem Wert $L_{xm} = \SI{41.9(2)}{\milli\henry}$ ab, der an der
Induktivitätsmessbrücke gemessen wird. Auch die Widerstände sind auf den fast gleichen Wert
bestimmt; die Abweichung des in Kapitel \ref{subsec:auswimb} bestimmten Wert von dem
in Kapitel \ref{sec:auswmaxw} beträgt \SI{1}{\percent}.

\subsection{Wien-Robinson-Brücke}

Der hier abgebildete Plot zeigt, dass die Messwerte sehr gut mit der Theoriekurve
übereinstimmen. Das Minimum ist bei $\frac{\nu}{\nu_0} = 1$ zu erkennen.
Außerdem wird $\nu_0$ aus $R$ und $C$ auf $\SI{240}{\hertz}$ bestimmt, genau wie
es aus den Messwerten zu erkennen ist.

\subsection{Klirrfaktor}

Der Klirrfaktor hat mit $\SI{0.58}{\percent}$ einen vernünftigen Wert.
