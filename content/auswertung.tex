\section{Auswertung}
\label{sec:Auswertung}

Zum Berechnen der einzelen Werte und des
Mittelwerts wurde
uncertainties \cite{uncertainties} benutzt. Hier wird der Wert
und der Fehler mit  der Formel nach Gauß'scher Fehlerfortpflanzung

\begin{equation}
    \phi = \sqrt{\sum_{i=1}^{N} \biggl(\frac{\partial f}{\partial x_i}\biggr)^2
    \cdot \sigma_i^2}
    \label{eqn:gauss}
\end{equation}

berechnet \eqref{eqn:gauss}.

\subsection{Wheatstonesche Brücke}

Mit Hilfe der Wheatstonesche Brücke wurden 2 unbekannte Widerstände bestimmt.
Die Formel, die verwendet wurde, ist die Folgende:

\begin{equation}
  R_x = R_2\frac{R_3}{R_4}.
  \label{eqn:Widerstand}
\end{equation}

Hierbei ist $R_x$ der unbekannte Widerstand, $R_2$ ist der bekannnte
Widerstand als festes Bauteil und $R_3$ und $R_4$ sind die mit dem
Potentiometer bestimmten Werte.
Zur Berechnung des Mittelwerts wurde die Formel \eqref{eqn:Rxm} benutzt.

\begin{equation}
  R_{xm} = \frac{1}{3}(R_{x1} + R_{x2} + R_{x3})
  \label{eqn:Rxm}
\end{equation}

\subsubsection{Messwerte und Ergebnis}

\paragraph{Bestimmung von Wert 13}

\begin{table}
  \centering
  \caption{Tabelle mit den Messdaten und dem Ergebnis $R_{xi}$.}
  \label{tab:Widerstand13}
  \begin{tabular}{c c c c}
    \toprule
    $R_2$ \ /\ \si{\Omega} & $R_3 \ /\ \si{\Omega}$ & $R_4 \ /\ \si{\Omega}$ & $R_x \ /\ \si{\Omega}$\\
    \midrule
    \num{332 +- 1} & \num{540 +- 3} & \num{460 +- 2} & \num{390 +- 3}\\
    \num{664 +- 1} & \num{369 +- 2} & \num{631 +- 3} & \num{388 +- 3}\\
    \num{1000 +- 2} & \num{279 +- 1} & \num{721 +- 4} & \num{387 +- 3}\\
    \bottomrule
  \end{tabular}
\end{table}

In Tabelle \ref{tab:Widerstand13} sind die Messdaten und die einzelnen
Ergebnisse bei der Bestimmung von Wert 13 aufgetragen.

Der Mittelwert beträgt dann $R_{xm} = \SI{319(1)}{\Omega}$.

\FloatBarrier
\paragraph{Bestimmung von Wert 12}

\begin{table}
  \centering
  \caption{Tabelle mit den Messdaten und dem Ergebnis $R_{xi}$.}
  \label{tab:Widerstand12}
  \begin{tabular}{c c c c}
    \toprule
    $R_2 \ /\ \si{\Omega}$ & $R_3 \ /\ \si{\Omega}$ & $R_4 \ /\ \si{\Omega}$ & $R_x \ /\ \si{\Omega}$\\
    \midrule
    \num{332 +- 1} & \num{540 +- 3} & \num{460 +- 2} & \num{390 +- 3}\\
    \num{664 +- 1} & \num{369 +- 2} & \num{631 +- 3} & \num{388 +- 3}\\
    \num{1000 +- 2} & \num{279 +- 1} & \num{721 +- 4} & \num{387 +- 3}\\
    \bottomrule
  \end{tabular}
\end{table}

In Tabelle \ref{tab:Widerstand12} sind die Messdaten und die einzelnen
Ergebnisse bei der Bestimmung von Wert 12 aufgetragen.

Der Mittelwert beträgt dann $R_{xm} = \SI{388(2)}{\Omega}$.

\subsection{Kapazitätsmessbrücke}

Mit Hilfe der Kapazitätsmessbrücke wurden die Kapazitäten
der zwei idealen Kondensatoren Wert 1 und 3 und des realen
Kondensators Wert 9 bestimmt.

Bei beiden Arten wird die Kapazität mit der Gleichung \eqref{eqn:kapazitaet}
bestimmt.

\begin{equation}
  C_x = C_2\frac{R_4}{R_3}.
  \label{eqn:kapazitaet}
\end{equation}

Hierbei ist $C_2$ ein Kondensator mit bekannter fester Kapazität. $R_3$ und
$R_4$ wurden mit dem Potentiometer bestimmt.
Bei den realen Kondensatoren wird  der Widerstand mit Formel
\eqref{eqn:Widerstand} berechnet.

Der Mittelwert des Widerstands wird wieder mit Gleichung \eqref{eqn:Rxm} berechnet.
Außerdem wird der Mittelwert der Kapazität mit Gleichung \eqref{eqn:Cxm} berechnet.

\begin{equation}
  Cxm = \frac{1}{3}(C_{x1} + C_{x2} + C_{x3})
  \label{eqn:Cxm}
\end{equation}

$C_{x1}$, $C_{x2}$ und $C_{x3}$ sind die einzeln ausgerechneten Kapazitäten der
jeweiligen Messung.

\subsubsection{Messwerte und Ergebnis}
\label{subsubsec:kapme}

\paragraph{Bestimmung von Wert 1}

\begin{table}
  \centering
  \caption{Tabelle mit den Messdaten und dem Ergebnis $C_{xi}$.}
  \label{tab:Kapazitaet1}
  \begin{tabular}{c c c c}
    \toprule
    $C_2 \ /\ 10^-7 \si{\farad}$ & $R_3 \ /\ \si{\Omega}$ & $R_4 \ /\ \si{\Omega}$ & $C_{xi} \ /\ 10^-7 \si{\farad}$\\
    \midrule
    \num{3.990 +- 0.008} & \num{375 +- 2} & \num{625 +- 3} & \num{6.65 +- 0.05}\\
    \num{7.50 +- 0.02} & \num{530 +- 3} & \num{470 +- 2} & \num{6.65 +- 0.05}\\
    \num{9.92 +- 0.02} & \num{601 +- 3} & \num{399 +- 2} & \num{6.59 +- 0.05}\\
    \bottomrule
  \end{tabular}
\end{table}

In Tabelle \ref{tab:Kapazitaet1} sind die Messdaten und die einzelnen
Ergebnisse bei der Bestimmung der Kapazität von Wert 1 aufgetragen.

Der Mittelwert beträgt dann $C_{xm} = \SI{6.63(3)e-7}{\farad}$.

\paragraph{Bestimmung von Wert 3}

\begin{table}
  \centering
  \caption{Tabelle mit den Messdaten und dem Ergebnis $C_{xi}$.}
  \label{tab:Kapazitaet3}
  \begin{tabular}{c c c c}
    \toprule
    $C_2 \ /\ 10^-7 \si{\farad}$ & $R_3 \ /\ \si{\Omega}$ & $R_4 \ /\ \si{\Omega}$ & $C_{xi} \ /\ 10^-7 \si{\farad}$\\
    \midrule
    \num{3.990 +- 0.008} & \num{486 +- 2} & \num{514 +- 3} & \num{4.22 +- 0.03}\\
    \num{7.50 +- 0.02} & \num{640 +- 3} & \num{360 +- 2} & \num{4.22 +- 0.03}\\
    \num{9.92 +- 0.02} & \num{703 +- 4} & \num{279 +- 2} & \num{4.19 +- 0.03}\\
    \bottomrule
  \end{tabular}
\end{table}

In Tabelle \ref{tab:Kapazitaet3} sind die Messdaten und die einzelnen
Ergebnisse bei der Bestimmung der Kapazität von Wert 1 aufgetragen.

Der Mittelwert beträgt dann $C_{xm} = \SI{4.21(2)e-7}{\farad}$.

\paragraph{Bestimmung von Wert 9}

\begin{table}
  \centering
  \caption{Tabelle mit den Messdaten und den Ergebnissen $C_{xi}$ und $R_{xi}$.}
  \label{tab:Kapazitaet9}
  \begin{tabular}{c c c c c c}
    \toprule
    $C_2 \ /\ 10^-7 \si{\farad}$ & $R_2 \ /\ \si{\Omega}$ & $R_3 \ /\ \si{\Omega}$ &
    $R_4 \ /\ \si{\Omega}$ & $C_{xi} \ /\ 10^-7 \si{\farad}$ & $R_{xi} \ /\ \si{\Omega}$ \\
    \midrule
    \num{3.990 +- 0.008} & \num{512 +- 15} & \num{479 +- 2} & \num{521 +- 3} & \num{4.34 +- 0.03} & \num{471 +- 15}\\
    \num{7.50 +- 0.02} & \num{270 +- 8} & \num{633 +- 3} & \num{367 +- 2} & \num{4.35 +- 0.03} & \num{466 +- 14}\\
    \num{9.92 +- 0.02} & \num{207 +- 6} & \num{694 +- 4} & \num{306 +- 2} & \num{4.37 +- 0.03} & \num{470 +- 15}\\
    \bottomrule
  \end{tabular}
\end{table}

In Tabelle \ref{tab:Kapazitaet9} sind die Messdaten und die einzelnen
Ergebnisse bei der Bestimmung der Kapazität und des Widerstands
von Wert 9 aufgetragen.

Der Mittelwert der Kapazität beträgt dann $C_{xm} = \SI{4.21(2)e-7}{\farad}$.
Außerdem beträgt der Mittelwert des Widerstands $R_{xm} =
\SI{469(8)}{\Omega}$.

\subsection{Induktivitätsmessbrücke}
\label{subsec:auswimb}

Die Induktivität und der Widerstand einer Spule wird mit Hilfe der Induktivitätsmessbrücke
bestimmt.

Zur Bestimmung des Widerstands wird erneut die Gleichung \eqref{eqn:Widerstand}
verwendet. $R_3$ und $R_4$ sind weiterhin die per Potentiometer bestimmten Werte.
Nun ist $R_2$ aber ein variabler Widerstand. Der Mittelwert wird hier ebenfalls
mit Gleichung \eqref{eqn:Rxm} berechnet.

Zur Berechnung der Induktivität wird die Gleichung \eqref{eqn:induk} verwendet.
Es gilt

\begin{equation}
  L_x = L_2\frac{R_3}{R_4}.
  \label{eqn:induk}
\end{equation}

$L_2$ ist die Spule mit bekannter Induktivität. Der Mittelwert der Induktivität wird
mit Gleichung \eqref{eqn:Lxm} bestimmt.

Es gilt:

\begin{equation}
  Lxm = \frac{1}{3}(L_{x1} + L_{x2} + L_{x3}).
  \label{eqn:Lxm}
\end{equation}

\subsubsection{Messwerte und Ergebnis}

\begin{table}
  \centering
  \caption{Tabelle mit den Messdaten und den Ergebnissen $L_{xi}$ und $R_{xi}$.}
  \label{tab:Induk17}
  \begin{tabular}{c c c c c c}
    \toprule
    $L_2 \ /\ \si{\milli\henry}$ & $R_2 \ /\ \si{\Omega}$ & $R_3 \ /\ \si{\Omega}$ &
    $R_4 \ /\ \si{\Omega}$ & $L_{xi} \ /\ \si{\milli\henry}$ & $R_{xi} \ /\ \si{\Omega}$ \\
    \midrule
    \num{14.60 +- 0.03} & \num{34 +- 1} & \num{741 +- 4} & \num{259 +- 1} & \num{41.8 +- 0.3} & \num{97 +- 3}\\
    \num{20.10 +- 0.04} & \num{41 +- 1} & \num{676 +- 3} & \num{324 +- 2} & \num{41.9 +- 0.3} & \num{86 +- 3}\\
    \bottomrule
  \end{tabular}
\end{table}

In Tabelle \ref{tab:Induk17} sind die Messdaten und die einzelnen
Ergebnisse bei der Bestimmung der Induktivität und des Widerstands
von der Spule Wert 17 aufgetragen.

Der Mittelwert der Induktivität beträgt nach Gleichung \eqref{eqn:Lxm}
$L_{xm} = \SI{41.9(2)}{\milli\henry}.$

Der Mittelwert des Widerstands beträgt nach Gleichung \eqref{eqn:Rxm}
$R_{xm} = \SI{91(2)}{\Omega}.$

\subsection{Maxwell-Brücke}

Darauf folgend wird die Induktivität und der Widerstand von Spule Wert 17 erneut
bestimmt. Diesmal aber mit Hilfe der Maxwell-Brücke.

Die Widerstandswerte weden wie in Kapitel \ref{subsec:auswimb} mit den Gleichungen
\eqref{eqn:Widerstand} und \eqref{eqn:Rxm} berechnet.

Die Induktivität wird mit der Gleichung \eqref{eqn:indukmax} bestimmt.
Es gilt:

\begin{equation}
  L_x = R_2 R_3 C_4.
  \label{eqn:indukmax}
\end{equation}

Hier ist $R_3$ einer der beiden variablen Widerstände in der Schaltung. $R_2$
ist ein bekannter fester Widerstand und $C_4$ ist ein Kondensator mit bekannter
fester Kapazität.
Der Mittelwert der Induktivität wird mit Gleichung \eqref{eqn:Lxm} berechnet.

\subsubsection{Messwerte und Ergebnis}

\begin{table}
  \centering
  \caption{Tabelle mit den Messdaten und den Ergebnissen $L_{xi}$ und $R_{xi}$.}
  \label{tab:Indukmax17}
  \begin{tabular}{c c c c c c}
    \toprule
    $C_4 \ /\ \si{\nano\farad}$ & $R_2 \ /\ \si{\Omega}$ & $R_3 \ /\ \si{\Omega}$ &
    $R_4 \ /\ \si{\Omega}$ & $L_{xi} \ /\ \si{\milli\henry}$ & $R_{xi} \ /\ \si{\Omega}$ \\
    \midrule
    \num{750 +- 2} & \num{332 +- 1} & \num{170 +- 5} & \num{612 +- 18} & \num{42 +- 1} & \num{92 +- 4}\\
    \num{750 +- 2} & \num{664 +- 1} & \num{84 +- 3} & \num{614 +- 18} & \num{42 +- 1} & \num{91 +- 4}\\
    \num{750 +- 2} & \num{1000 +- 2} & \num{55 +- 2} & \num{626 +- 19} & \num{41 +- 1} & \num{88 +- 4}\\
    \bottomrule
  \end{tabular}
\end{table}

Der Mittelwert der Induktivität beträgt nach Gleichung \eqref{eqn:Lxm}
$L_{xm} = \SI{41.8(7)}{\milli\henry}.$

Der Mittelwert des Widerstands beträgt nach Gleichung \eqref{eqn:Rxm}
$R_{xm} = \SI{90(2)}{\Omega}.$

\subsection{Wien-Robinson-Brücke}

\subsubsection{Messwerte}

\begin{table}
  \centering
  \caption{Tabelle mit den Messdaten der Wien-Robinson-Brücke.}
  \label{tab:frequenz}
  \begin{tabular}{c c}
    \toprule
    $\nu \ /\ \si{\hertz}$ & $U_{Br} \ /\ \si{\milli\volt}$\\
    \midrule
    20 & 1450\\
    40 & 1350\\
    60 & 1200\\
    80 & 1030\\
    100 & 860\\
    110 & 780\\
    120 & 700\\
    130 & 620\\
    140 & 550\\
    150 & 475\\
    160 & 410\\
    170 & 350\\
    180 & 295\\
    190 & 240\\
    200 & 190\\
    210 & 140\\
    220 & 98\\
    230 & 53\\
    240 & 15\\
    250 & 47\\
    260 & 87\\
    270 & 125\\
    280 & 155\\
    290 & 190\\
    300 & 220\\
    400 & 508\\
    500 & 708\\
    1000 & 1180\\
    2000 & 1300\\
    5000 & 1430\\
    10000 & 1440\\
    30000 & 1440\\
    \bottomrule
  \end{tabular}
\end{table}

In Tabelle \ref{tab:frequenz} sind die Messwerte aus der Messung an der
Wien-Robinson-Brücke aufgetragen.

Außerdem wurde $U_S$ auf \SI{4.36}{\volt} bestimmt. $R$ war \SI{1}{\kilo\Omega}
groß. Als Kondensator wurde Wert 1 benutzt, dessen Kapazität $C_{xm}$ in Kapitel
\ref{subsubsec:kapme} auf gerundete \SI{663}{\nano\farad} bestimmt wurde.

\FloatBarrier

\subsubsection{Rechnung und Plot}

\begin{figure}[h]
  \centering
  \includegraphics[width = \textwidth]{build/plote.pdf}
  \caption{Plot der Messwerte sowie der Theoriekurve.}
  \label{fig:WRBplot}
\end{figure}

In Abbildung \ref{fig:WRBplot} sind die Messwerte $U_{BR}(\nu) \ /\ U_S$ gegen
$\nu \ /\ \nu_0$ aufgetragen. Außerdem ist dort die Theoriekurve nach Gleichung
\eqref{eqn:WRBG} eingezeichnet. Der Plot wurde mit dem package Matplotlib aus
python erstellt \cite{matplotlib}.

Zu bestimmen war außerdem das $\nu_0$ bei dem die Brückenspannung minimal wird.

Dies erfolgt mit Gleichung \eqref{eqn:nu0}:

\begin{equation}
  \nu_0 = \frac{1}{2 \pi R C}.
  \label{eqn:nu0}
\end{equation}

Das Ergebnis ist dann: $\nu_0 = 240$.

\subsection{Klirrfaktor}

Der Klirrfaktor wird mit der Gleichung \eqref{eqn:klirrf} berechnet.
$U_2$ wird mit

$U_2 = \frac{U_{Br}}{f(2)}$ bestimmt.

In diesem Fall : $U_2 = \frac{\SI{15}{\milli\volt}}{\frac{9}{73}} =
\SI{122}{\milli\volt}$.

Somit ist der Klirrfaktor:\\
\\$k = \frac{\SI{122}{\milli\volt}}{\SI{1450}{\milli\volt}} = \SI{8.4}{\percent}$.
