\newpage

 \section{Durchführung}
\label{sec:Durchführung}

Als Stromquelle werde in allen Teilversuchen ein
Sinus-Wechselspannungs-Generator benutzt. In den Teilversuchen \ref{sec:D1}
bis \ref{sec:D4} benutze man als Frequenz $\SI{800}{\Hz}$.
Als Nulldetektor, also als Messgerät, benutze man ein Oszilloskop. Daran
kann man gegebenenfalls einen Tiefpass anschliessen um starke Störspannungen
des Generators auszugleichen.
Für die variablen Widerstände benutze man Zehngang-Präzisions-Potentiometer
mit einem Maximalwiderstand von $\SI{1}{\kilo\ohm}$.


\subsection{Wheatstonesche Brücke}
\label{sec:D1}

Man messe im ersten Versuch ohmsche Widerstände mit Hilfe der Wheatstoeschen
Brückenschaltung. Dazu gehe man wie folgt vor:

\begin{enumerate}

\item Man baue die Schaltung in Abbildung \ref{fig:WheBr} auf.

\item Man verändere die variablen Widerstände $R_3$ und $R_4$ am Potentiometer
so, dass das Oszilloskop eine Spannungs-Amplitude nahe Null anzeigt.
Dabei muss im Wechsel die Empfindlichkeit des Oszilloskops hochgestellt
und am Potentiometer variiert werden. Man notiere den Wert für $R_3$.

\item Insgesamt sollen 6 Messungen durchgeführt werden, wobei zwei unbekannte
Widerstände $R_x$ und jeweils drei bekannte Widerstände $R_2$ benutzt werden.

\end{enumerate}


\subsection{Kapazitätsmessbrücke}

Mann messe im zweiten Versuch ideale und reale Kapazitäten mit Hilfe
der Kapazitätsmessbrücke. Dazu gehe man wie folgt vor:

\begin{enumerate}

\item Man baue für die Messung einer idealen Kapazität die Schaltung in Abbildung
\ref{fig:KapBr} ohne die Verlust-Widerstände $R_x$ und $R_2$ auf.

\item Man verändere die Widerstände am Potentiometer erneut so, dass die angezeigte
Amplitude Null wird und gehe dabei genauso vor wie im vorherigen Versuch.

\item Insgesamt sollen 6 Messungen für ideale Kondensatoren durchgeführt werden.
Dafür werden zwei unbekannte Kapazitäten $C_x$ und jeweils drei
bekannte Kapazitäten $C_2$ benutzt.

\item Für die Messung einer realen Kapazität baue man die Schaltung in Abbildung
\ref{fig:KapBr} vollständig auf.

\item Man verändere nun im Wechsel sowohl die variablen Widerstände $R_3$, $R_4$
und $R_2$ als auch die Empfindlichkeit des Oszillographen.
An den Potentiometern soll jeweils so lange varriiert werden, bis ein Minimum
der Amplitude am Oszillographen erkennbar ist. Dies wiederholt man,
bis sich eine Amplitude nahe Null einstellt.
Man notiere sowohl den Wert für $R_3$, als auch den für $R_2$.

\item Insgesamt führe man 3 Messungen für reale Kondensatoren durch. Dabei
benutze man einen unbekannten Kondensatoren $C_x$ und drei bekannte $C_2$.

\end{enumerate}


\subsection{Induktivitätsmessbrücke}

Im dritten Versuch messe man reale Induktivitäten mit Hilfe
der Induktivitätsmessbrücke. Dazu gehe man wie folgt vor:

\begin{enumerate}

\item Man baue zur Messung einer realen Induktivität die Schaltung in Abbildung
\ref{fig:IndBr} auf.

\item Man gehe nun genauso vor wie bei der Messung eines realen Kodensators,
variiert also die Widerstände an beiden Potentiometern und erhöht die
Empfindlichkeit des Oszilloskops bis die Amplitude etwa Null ist.

\item Man führe insgesamt 2 Messungen durch, wobei eine unbekannte reale
Spule und zwei bekannte reale Spulen verwendet werden.

\end{enumerate}


\subsection{Maxwell-Brücke}
\label{sec:D4}

Im vierten Versuch messe man erneut reale Induktivitäten. Dafür benutze man
dieses Mal die Maxwellsche Brückenschaltung. Dazu gehe man wie folgt vor:

\begin{enumerate}

\item Man baue zunächst die Schaltung in Abbildung \ref{fig:MaxBr} auf.

\item Man verändere nun, sobald Minima erkennbar sind, im Wechsel die variablen
Widerstände $R_3$ und $R_4$ und die Empfindlichkeit des Oszilloskops
bis die Amplitude etwa Null ist. In diesem Versuch notiere man die Werte
für die Widerstände $R_3$ und $R_4$.

\item Man messe insegesamt eine unbekannte reale Spule mit
insgesamt drei verschiedenen Widerständen $R_2$, hat also am Ende 3
verschiedene Messwertpaare.

\end{enumerate}

\subsection{Wien-Robinson-Brücke}

Im fünften und letzten Versuch messe man die Abhängigkeit zwischen der
Frequenz und der Brückenspannung mit Hilfe der Wien-Robinson-Brücke.
Dazu befolge man folgende Schritte:

\begin{enumerate}

\item Dazu baue man zunächst die Schaltung in Abbildung \ref{fig:WRBr} auf.

\item Hier muss man in der Brückenschaltung selbst nichts verändern. Man
verändere hier die Frequenz des Sinus-Generators und nehme am Oszilloskop
die Werte für die Amplitude auf. Man sollte dabei die Empfindlichkeit des
Oszilloskops verändern um genaue Messwerte aufzunehmen. Man notiere sich
die Frequenz $\omega$ und die Brückenspannung $U_B$. Insgesamt sollte man
etwa 30 Messwerte aufnehmen.

\item Zuletzt messe man die Speisespanung $U_S$ einmalig direkt an der
Stromquelle.

\end{enumerate}
