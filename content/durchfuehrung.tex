\newpage

 \section{Durchführung}
\label{sec:Durchführung}

Als Stromquelle werde in allen Teilversuchen ein
Sinus-Wechselspannungs-Generator benutzt. In den Teilversuchen \ref{sec:D1}
bis \ref{sec:D4} wird als Frequenz $\SI{800}{\Hz}$ benutzt.
Als Nulldetektor, also als Messgerät, wird ein Oszilloskop benutzt. Daran
kann gegebenenfalls ein Tiefpass angeschlossen werden um starke Störspannungen
des Generators auszugleichen.
Für die variablen Widerstände werde ein Zehngang-Präzisions-Potentiometer
mit einem Maximalwiderstand von $\SI{1}{\kilo\ohm}$ benutzt.


\subsection{Wheatstonesche Brücke}
\label{sec:D1}

Im ersten Versuch werden ohmsche Widerstände mit Hilfe der Wheatstoneschen
Brückenschaltung gemessen. Dazu wird wie folgt vorgegangen:

\begin{enumerate}

\item Zunächst wird die Schaltung in Abbildung \ref{fig:WheBr} aufgebaut.

\item Die variablen Widerstände $R_3$ und $R_4$ am Potentiometer werden so
verändert, dass das Oszilloskop eine Spannungs-Amplitude nahe Null anzeigt.
Dabei muss im Wechsel die Empfindlichkeit des Oszilloskops hochgestellt
und am Potentiometer variiert werden. Der Wert für $R_3$ wird notiert.

\item Insgesamt sollen 6 Messungen durchgeführt werden, wobei zwei unbekannte
Widerstände $R_x$ und jeweils drei bekannte Widerstände $R_2$ benutzt werden.

\end{enumerate}


\subsection{Kapazitätsmessbrücke}

Im zweiten Versuch werden ideale und reale Kapazitäten mit Hilfe
der Kapazitätsmessbrücke gemessen. Dazu wird wie folgt vorgegangen:

\begin{enumerate}

\item Für die Messung einer idealen Kapazität wird die Schaltung in Abbildung
\ref{fig:KapBr} ohne die Verlust-Widerstände $R_x$ und $R_2$ aufgebaut.

\item Die Widerstände am Potentiometer werden erneut so verändert, dass die
angezeigte Amplitude Null wird.

\item Insgesamt sollen 6 Messungen für ideale Kondensatoren durchgeführt werden.
Dafür werden zwei unbekannte Kapazitäten $C_x$ und jeweils drei
bekannte Kapazitäten $C_2$ benutzt.

\item Für die Messung einer realen Kapazität wird die Schaltung in Abbildung
\ref{fig:KapBr} vollständig aufgebaut.

\item Nun wird im Wechsel sowohl die variablen Widerstände $R_3$, $R_4$
und $R_2$ als auch die Empfindlichkeit des Oszillographen verändert.
An den Potentiometern soll jeweils so lange varriiert werden, bis ein Minimum
der Amplitude am Oszillographen erkennbar ist. Dies wird wiederholt,
bis sich eine Amplitude nahe Null einstellt.
Sowohl der Wert für $R_3$, als auch der für $R_2$ wird notiert.

\item Insgesamt werden 3 Messungen für reale Kondensatoren durchgeführt. Dabei
wird ein unbekannter Kondensator $C_x$ und drei bekannte $C_2$ benutzt.

\end{enumerate}


\subsection{Induktivitätsmessbrücke}

Im dritten Versuch werden reale Induktivitäten mit Hilfe
der Induktivitätsmessbrücke gemessen. Dazu wird wie folgt vorgegangen:

\begin{enumerate}

\item Zur Messung einer realen Induktivität wird die Schaltung in Abbildung
\ref{fig:IndBr} aufgebaut.

\item Es wird nun genauso vorgegangen wie bei der Messung eines realen
Kodensators, also werden die Widerstände an beiden Potentiometern varriiert und
die Empfindlichkeit des Oszilloskops erhöht bis die Amplitude etwa Null ist.

\item Insgesamt werden 2 Messungen durchgeführt, wobei eine unbekannte reale
Spule und zwei bekannte reale Spulen verwendet werden.

\end{enumerate}


\subsection{Maxwell-Brücke}
\label{sec:D4}

Im vierten Versuch werden erneut reale Induktivitäten gemessen. Dafür wird
dieses Mal die Maxwellsche Brückenschaltung benutzt.
Es wird wie folgt vorgegangen:

\begin{enumerate}

\item Zunächst wird die Schaltung in Abbildung \ref{fig:MaxBr} aufgebaut.

\item Nun werden, sobald Minima erkennbar sind, im Wechsel die variablen
Widerstände $R_3$ und $R_4$ und die Empfindlichkeit des Oszilloskops verändert
bis die Amplitude etwa Null ist. In diesem Versuch werden die Werte
für die Widerstände $R_3$ und $R_4$ notiert.

\item Insegesamt werden eine unbekannte reale Spule mit
insgesamt drei verschiedenen Widerständen $R_2$ gemessen. Also werden 3
verschiedene Messwertpaare erhalten.

\end{enumerate}

\subsection{Wien-Robinson-Brücke}

Im fünften und letzten Versuch wird die Abhängigkeit zwischen der
Frequenz und der Brückenspannung mit Hilfe der Wien-Robinson-Brücke gemessen.
Dazu werden folgende Schritte befolgt:

\begin{enumerate}

\item Zunächst wird die Schaltung in Abbildung \ref{fig:WRBr} aufgebaut.

\item Hier muss in der Brückenschaltung nichts verändert werden. Hier wird
die Frequenz des Sinus-Generators verändert und am Oszilloskop
die Werte für die Amplitude aufgenommen. Die Empfindlichkeit
des Oszilloskops soll dabei verändert werden um genaue Messwerte aufzunehmen.
Es werden die Frequenz $\omega$ und die Brückenspannung $U_B$ notiert.
Insgesamt werden etwa 30 Messwerte aufgenommen.

\item Zuletzt wird die Speisespanung $U_S$ einmalig direkt an der
Stromquelle gemessen.

\end{enumerate}
