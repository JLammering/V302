\section{Diskussion}
\label{sec:Diskussion}

\subsection{Wheatstonesche Brücke}

Die bestimmten Widerstände aus den einzelnen Messungen liegen, wenn man die berechnete
Unsicherheit mit einbezieht, dicht aneinander. Auch der Mittelwert liegt in den jeweiligen
Fehlerintervallen. Dies ist auch auf die verhältnismäßig geringen Unsicherheiten der
der Bauteile zurückzuführen.

\subsection{Kapazitätsmessbrücke}

\paragraph{Ideale Kondensatoren}
Hier sind jeweils die ersten beiden Werte identisch und der Dritte weicht etwas ab.
Aber auch der ist noch im Rahmen des Fehlerintervalls. Somit ist der Mittelwert
auch in guter Übereinstimmung mit den Mittelwerten. Erneut hatten die Bauteile
keine große Unsicherheit.

\paragraph{Reale Kondensatoren}
Bei der Bestimmung der Kapazität des realen Kondensators treten ähnlich große Abweichungen
auf, die allesamt im Rahmen der Messgenauigkeit liegen.

Durch die hohe Unsicherheit von $R_2$ mit \SI{3}{\percent} kann der Widerstand
nur sehr ungenau bestimmt werden. Offensichtlich ist diese Schaltung also nicht
besonders geeignet um den Widerstand eines Kondensators zu bestimmen.

\subsection{Induktivitätsmessbrücke}

Die Induktivität konnte hier nur mit einer Unsicherheit bestimmt werden, die
um eine Größenordung größer ist als die der festen Bauteile. Dies liegt daran, dass
$R_2$ wieder eine hohe relative Unsicherheit hatte. Beim Widerstand ist dies analog
der Fall.

\subsection{Maxwell-Brücke}

Nun wurde dieselbe Spule erneut vermessen, nun mit der Maxwellbrücke. Die Werte für
die Induktivität sind sich mit $L_{xm} = \SI{41.9(2)}{\milli\henry}$ bei der
Induktivitätsmessbrücke und $L_{xm} = \SI{41.8(7)}{\milli\henry}$  bei der
Maxwell-Brücke sehr ähnlich. Auch die Widerstände wurden auf den fast gleichen Wert
bestimmt.

\subsection{Wien-Robinson-Brücke}

Der hier abgebildete Plot zeigt, dass die Messwerte sehr gut mit der Theoriekurve
übereinstimmen. Das Minimum ist bei $\frac{\nu}{\nu_0} = 1$ zu erkennen.
Außerdem wurde $\nu_0$ aus $R$ und $C$ auf $\SI{240}{\hertz}$ bestimmt, genau wie
es aus den Messwerten zu erkennen ist.

\subsection{Klirrfaktor}

Der Klirrfaktor hat mit $\SI{8.4}{\percent}$ einen vernünftigen Wert.
